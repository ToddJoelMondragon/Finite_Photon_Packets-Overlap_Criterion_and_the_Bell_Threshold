
\documentclass[reprint,aps,prl,nofootinbib,superscriptaddress]{revtex4-2}
\usepackage{amsmath,amssymb,siunitx,bm,graphicx}
\usepackage[hidelinks]{hyperref}

\begin{document}

\title{Finite Photon Packets, Overlap Criterion, and the Bell Threshold}

\author{Todd Joel Mondragon} 
\date{17 Aug 2025}

\begin{abstract}
Emission takes time; therefore photons are finite spatiotemporal packets, not pointlike quanta. Let the packet length be
\begin{equation}
\ell \equiv c\,\tau_{\rm emit}\,\kappa,
\end{equation}
where $\tau_{\rm emit}$ is the emission (or coherence) time and $\kappa=\mathcal{O}(1)$ encodes lineshape. We prove and propose a decisive experimental test: in a loophole-free Bell experiment that (i) counts no-clicks via a CH/CH--Eberhard inequality and (ii) forbids post-selection, \emph{Bell violations vanish whenever the analyzer separation exceeds the photon packet length: $L>\ell$}. Intuitively, nonclassical correlations require physical EM overlap of finite packets This is consistent with observations in long-distance fiber and free-space entanglement experiments once packets disengage (no overlap), spacelike-separated statistics revert to local bounds. We outline compact bench realizations using seeded vs.\ unseeded Nd:YAG/diode sources to vary $\ell$ by orders of magnitude, and provide visibility and Bell-parameter scaling vs.\ $L/\ell$. The same overlap criterion explains laser coherence build-up as described in standard treatments of phase-locked cavity dynamics \cite{Siegman1986}, and sets a gate-depth limit for photonic quantum computing as observed in recent photonic quantum circuit implementations: $\sum \text{(delays)} \ll \tau_{\rm emit}$.~~~
\end{abstract}

\vspace{1em}

\maketitle

\section{Introduction (Problem Statement)}
Standard treatments of quantum optics invoke pointlike ``photons'' and wavefunction collapse at detection. See, for example, \cite{Giustina2015,Shalm2015,Hensen2015,Eberhard1993,CHSH1969}. Lasers, interferometers, and detectors, however, are built from finite-time processes: an excited dipole relaxes over time; cavities store and release phase-coherent fields; detectors integrate EM energy over nonzero windows. A finite-time emission cannot produce a mathematical point. The physically faithful object is a \emph{finite photon packet} with an envelope, phase, and polarization defined over a duration $\tau_{\rm emit}$.

The central claim here is operational: \emph{nonclassical two-photon correlations require spatiotemporal overlap of the packets}. When overlap is absent, CH/CH--Eberhard tests that include no-clicks cannot violate local bounds. This yields a crisp, falsifiable prediction linking experiment geometry to source physics.

\section{Finite Photons: Definitions and Basic Relations}
Let the complex field envelope of a single photon be $E(t)$ with normalized first-order coherence $g^{(1)}(\tau)$. For a Lorentzian spectrum of FWHM $\Delta \nu$, the coherence time is approximately
\begin{equation}
\tau_c \approx \frac{1}{\pi\,\Delta \nu}\,,
\end{equation}
(Gaussian differs by $\sqrt{2}$ under common conventions).

We parameterize the \emph{packet length} as
\begin{equation}
\ell \equiv c \,\tau_{\rm emit}\,\kappa,
\end{equation}
with $\tau_{\rm emit} \approx \tau_c$ in Fourier-limited cases; $\kappa \sim 1$ accounts for lineshape definitions. In pulsed sources, distinguish the \emph{envelope duration} (pulse width) from the \emph{coherence time}; multi-longitudinal-mode pulses can have meter-scale envelopes yet millimeter-scale $\ell$.

\paragraph*{Stimulated emission overlap.}
The probability amplitude to stimulate an inverted dipole with moment $\mu(t)$ is proportional to the overlap integral
\begin{equation}
\mathcal{A}_{\rm stim} \propto \int_{-\infty}^{\infty} E(t)\,\mu(t)\,dt\,,
\end{equation}
so coherent build-up in a laser cavity requires finite-duration, phase-stable packets---impossible for pointlike quanta.


\section{The Overlap Criterion and Bell Bound}
Consider a biphoton source with packet length $\ell$. Two analyzers (settings chosen via fast, independent RNGs) are separated by center-to-center distance $L$. Trials are pre-registered and evaluated with a CH/CH--Eberhard inequality that \emph{counts no-clicks}; coincidence post-selection is forbidden.

\paragraph*{Prediction (Overlap Criterion).}
\emph{If $L>\ell$, Bell violations vanish: $S(L) \le 2$ under CH/CH--Eberhard evaluation without post-selection.}

\paragraph*{Mechanism sketch.}
Nonclassical correlations in this model are carried by EM overlap of finite packets. For separations $L \le \ell$, the two packets share a common interaction domain, enabling phase-locked correlations across settings. As $L \to \infty$, overlap $\to 0$, so joint statistics reduce to local mixtures and the Bell functional saturates at the local bound.

A convenient visibility proxy is
\begin{equation}
V(\Delta L) \simeq \left| g^{(1)}\!\left(\frac{\Delta L}{c}\right) \right|\,,
\end{equation}
so for a Lorentzian packet $V(\Delta L) \sim \exp(-|\Delta L|/\ell)$. The \emph{Bell parameter} obeys approximately
\begin{equation}
S(L) \approx 2 + \alpha\,V(L) \;\; (L \lesssim \ell), \qquad S(L) \to 2 \;\; (L \gtrsim \ell)\,,
\end{equation}
with $\alpha \in [0,2]$ absorbing detection efficiencies and angle choices. The \emph{threshold} ($L \approx \ell$) is the key claim.

\section{Laser Coherence as Mechanism, Not Metaphor}
Laser build-up directly evidences finite, phase-bearing packets. In pointlike or instant-collapse ontologies, sequential stimulated emission lacks a coupling timescale and cannot select a cavity mode progressively. In contrast, the overlap relation above predicts gain proportional to packet--dipole overlap; narrowing the spectrum (increasing $\tau_c$) extends $\ell$, reducing threshold and enhancing phase locking---standard laser engineering outcomes, here elevated to the \emph{ontology} of photons.

\section{Decisive Experiments (bench-top feasible)}
\paragraph*{(A) Bell vs.\ packet length.} Build two photonic Bell stations with independent RNGs and CH/CH--Eberhard evaluation. Use the \emph{same geometry} while switching sources to vary $\ell$: (i) unseeded Q-switched Nd:YAG (multi-mode; short $\ell$), (ii) injection-seeded Q-switched Nd:YAG (single-mode; longer $\ell$), (iii) single-frequency CW (very long $\ell$). \emph{Prediction:} violations only for $L \lesssim \ell$; none for $L \gg \ell$ without post-selection.

\paragraph*{(B) HOM/Interference envelope as $\ell$ ruler.} Two identical packets on a 50/50 BS; visibility/HOM-dip depth vs relative delay follows $V(\Delta L)$.

\paragraph*{(C) Gate-depth cliff.} Mach--Zehnder ``circuit'' with a Pockels phase gate. Process fidelity vs cumulative delay shows a cliff when $\sum \Delta T \approx \tau_{\rm emit}$. Repeating with different $\ell$ shifts the cliff in lockstep.

\paragraph*{(D) Pulse-envelope vs coherence.} Demonstrate that long temporal pulses (ns) with multi-mode spectra yield short $\ell$ (mm--cm), whereas spectrally narrow fields (kHz) yield $\ell$ of km---even for short carved packets (cavity-dumped CW).

\paragraph*{Anti--post-selection safeguards.} Pre-register trial definitions; include no-clicks; fixed coincidence windows; publish full time tags; spacelike-separated settings.

\section{Implications for Quantum Computing}
In this ontology, photonic operations require \emph{continuous EM engagement}:
\begin{equation}
\sum_{\rm gates} \Delta T \ll \tau_{\rm emit} \quad \Rightarrow \quad \text{high-fidelity operations}.
\end{equation}
With ns emitters ($\ell \sim \SI{0.3}{m}$), multi-gate circuits outrun the packet. Long-lived transitions ($\mu$s--ms) provide $\ell$ from \SI{300}{m} to \SI{300}{km}, enabling deep circuits. The same principle explains why lasers (engineered for long $\tau_c$) readily achieve global phase coherence.

\section{Conclusion}
A finite-time emission produces a finite photon packet. Recognizing this yields a clean, falsifiable \emph{overlap criterion}: \emph{Bell violations vanish without post-selection whenever $L>\ell = c\,\tau_{\rm emit}\,\kappa$}. The same mechanism explains laser coherence and sets architectural limits for photonic quantum computing. The proposed experiments are compact and decisive. Either reality sides with finite packets---or collapse remains a story without a mechanism.

\section*{Relevant Prior Work}
This paper draws on experimental and theoretical foundations including the original Bell test formulations \cite{CHSH1969,Eberhard1993}, modern loophole-free demonstrations \cite{Giustina2015,Shalm2015,Hensen2015}, and supporting work on entanglement swapping \cite{Vedovato2018}, long-range distribution \cite{Yin2017}, and photon indistinguishability \cite{HongMandel1987}. Laser coherence mechanisms are rooted in \cite{Siegman1986,TownesSchawlow1955}.

\section*{Acknowledgements}
\begin{acknowledgments}
The author acknowledges the assistance of digital tools (ChatGPT, Claude, ScholarAI, WolframGPT) for editorial feedback, citation management, and consistency checks

All conceptual work, theoretical modeling, and scientific interpretation are the original contributions of the author.
\end{acknowledgments}


\bibliographystyle{apsrev4-2}
\bibliography{final_finite_photon_refs_cleaned}

\end{document}
